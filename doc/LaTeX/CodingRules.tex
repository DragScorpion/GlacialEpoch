\documentclass{scrartcl}

\usepackage[ngerman]{babel}
\usepackage[utf8]{inputenc}
\usepackage[T1]{fontenc}
\usepackage{lmodern}
\usepackage[breaklinks=true]{hyperref}
\usepackage{nameref}
\usepackage{underscore}

\title{Coding Rules}
\author{Hendrik Wagemann}
\date{\today}

\begin{document}

\maketitle
\tableofcontents

\section{Einleitung}
\label{introduction}
Wir sind zwar nur eine kleine Gruppe, dennoch sollte es ein paar kleine Regelungen geben.

\section{Regeln}
\label{rules}

\subsection{Namensgebung}
\label{rules_naming}
Alle hier angegebenen Namensregels sind erstmal nur Vorschläge von mir, Hendrik.
Alle Namen auf Englisch
\begin{table}
	\begin{tabular}{ l | l | l }
		\textbf{Typ} & \textbf{Namensgebung} & \textbf{Beispiel} \\ 
		\hline
		Dateien & UpperCamelCase mit Präfix ''GE_'' & GE_BaseItem \\ 
		public, not nested classes (''Normale'' Klassen) & gleich dem Dateinamen & GE_BaseItem \\ 
		sonstige Klassen &	UpperCamelCase & DasIstEinKlassenName \\ 
		Methoden & lowerCamelCase & dasIstEineMethode \\ 
		Variablen & lowerCamelCase & dasIstEineVariable \\ 
	\end{tabular}
	\label{rules_naming_table1}
	\caption{Namensgebung bei Dateien, Klassen, Methoden und Variablen}
\end{table}
\begin{table}
	\begin{tabular}{ l | l }
		Minecraft Items & Präfix ''item'' \\ 
		Minecraft Blöcke & Präfix ''block'' \\ 
	\end{tabular}
	\label{rules_naming_table2}
	\caption{Sonderfälle bei speziellen Variablen}
\end{table}
\end{document}